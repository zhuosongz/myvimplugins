" System vimrc file for MacVim
"
" Maintainer:	Bjorn Winckler <bjorn.winckler@gmail.com>
" Last Change:	Sat Aug 29 2009
source $VIMRUNTIME/vimrc_example.vim
source $VIMRUNTIME/mswin.vim
behave mswin

if has("gui_running")
    set encoding=utf-8
    set fileencodings=ucs-bom,utf-8,chinese,prc,taiwan,latin-1

    if has("win32")
        set fileencoding=chinese
    else
        set fileencoding=utf-8
    endif

    let &termencoding=&encoding

source $VIMRUNTIME/delmenu.vim
source $VIMRUNTIME/menu.vim
language messages zh_CN.utf-8
endif

set nu!                    "显示行号
"set wrap!                 "自动折行
"set noautoindent          "不自动对齐
set autoindent             "自动对齐
"全局性地关闭令人讨厌的模式高亮(特别是换行符):
set hls!
"文档的一行不超过110个字符
match DiffAdd '\%>110v.*'
"设置并开启拼写检查,设置语言为en。
"setlocal spell spelllang=en


set nolinebreak
set number
set so=1
set sidescrolloff=1
set sidescroll=1
set makeprg=make
set tabstop=4
set softtabstop=4
set shiftwidth=4
set fdc=2 
"set fdm=syntax
set fdm=marker "将折叠方式设为marker
syn on
compiler gcc
set guioptions-=T
"set guioptions-=r
"set guioptions-=L
"set guioptions+=i
"set guioptions-=m
"set guioptions+=c
"set! fileformat=unix
"set fileformats=unix
set cindent
set nocompatible
set expandtab
set smarttab
set linespace=1
set foldopen-=search " don't open folds when you search into them
set foldopen-=undo " don't open folds when you undo stuff
set fsync
set laststatus=2
"set statusline=%t%r%h%w\ [%Y]\ [%{&ff}]\ [%{&fenc}:%{&enc}]\ [%08.8L]\ [%p%%-%P]\ [%05.5b]\ [%04.4B]\ [%08.8l]%<\ [%04.4c-%04.4v%04.4V]
set nocursorline
set nocursorcolumn
set updatecount=819222
set undolevels=819222
"set history=819222
set nobackup! "不生成备份文件(~文件)。
set sessionoptions+=unix,slash
set fo=tcqmM
"即formatoptions。设置vim的文本格式,默认是tcq。t表示自动换行(不包括注
"释),c表示注释也自动换行,换行的时候会自动在新行前面补注释符,q表示注
"释可以使用gq命令来格式化。m表示可以在任何值高于255 的多字节字符上分行,
"这对亚洲文本尤其有用。M表示在连接行时,不要在多字节字符之前或之后插入
"空格。
set showmode
"显示工作模式。例如编辑的时候按i,在vim屏幕下方会显示– INSERT –表示你在
"插入状态。 
set ru 
"打开 VIM 的状态栏标尺。它能即时显示当前光标所在位置在文件中的行
"号、列号,以及对应的整个文件的百分比。
set laststatus=2 
"在vim屏幕下方显示状态栏。状态栏会显示文件名称、光标位置等信息。


"以下是编辑器的显示色彩配置
"方案一
"colorscheme carvedwood
"colorscheme slate
"colorscheme ps_color
"colorscheme cool "BEST
"colorscheme InkPot
"colorscheme xoria256
"colorscheme southernlights
syntax enable
set background=light
" colorscheme molokai
set relativenumber

" 关闭警告提示音
set noerrorbells
set vb t_vb=

"方案二
"colorscheme moria
" let moria_style = 'black'
" let moria_style = 'dark'
" colo moria

"方案三
" Color scheme at present. 
" if ! has("gui_running") 
"    set t_Co=256 
" endif 
" set background=light gives a different style, feel free to choose between " them. 
" set background=dark 
" colors peaksea 
" colors jlbcool


"设置latex suite
set shellslash
filetype indent on
filetype plugin on
filetype on
let g:tex_flavor='latex'
set grepprg=grep\ -nH\ $*
" end

" autocmd Filetype tex source D:\Practical\Vim\vimfiles\auctex.vim

" 设置缩进
set shiftwidth=4
"set tabstop=4


"自动检测文件类型
filetype on


"自动折行(软回车)
"set textwidth=109
"set wrap!

"自动换行(硬回车)
"set textwidth=49 
set nowrap
"set wrapmargin=10



imap ;o \omega 
imap ;v \phi 
imap ;i \vee 
imap ;1 \uparrow 
imap ;3 \downarrow 
imap ;4 \leftarrow 
imap ;5 \rightarrow 
let g:Tex_Leader = ';'

"execute pathogen#infect()
filetype plugin indent on
"Hello, World!"
	let g:loaded_vimballPlugin= 1
	let g:loaded_vimball      = 1
":call plug#begin('~/.vim/plugged')

" Make sure you use single quotes
" Add two eggs

"set nocompatible              " be iMproved, required
filetype off                  " required







set nocompatible

" The default for 'backspace' is very confusing to new users, so change it to a
" more sensible value.  Add "set backspace&" to your ~/.vimrc to reset it.
set backspace+=indent,eol,start

" Disable localized menus for now since only some items are translated (e.g.
" the entire MacVim menu is set up in a nib file which currently only is
" translated to English).
set langmenu=none

set nocompatible              " be iMproved, required
filetype off                  " required

" set the runtime path to include Vundle and initialize
set rtp+=~/.vim/bundle/Vundle.vim
call vundle#begin()
" alternatively, pass a path where Vundle should install plugins
"call vundle#begin('~/some/path/here')

" let Vundle manage Vundle, required
Plugin 'VundleVim/Vundle.vim'

"
"
"
" Plugins 
"Plugin 'Vimball'
Plugin 'scrooloose/nerdtree'


" open a NERDTree automatically when vim starts up if no files were specified
autocmd StdinReadPre * let s:std_in=1
autocmd VimEnter * if argc() == 0 && !exists("s:std_in") | NERDTree | endif





" Stick this in your vimrc to open NERDTree with Ctrl+n (you can set whatever
" key you want):
map <C-g> :NERDTreeToggle<CR>



" close vim if the only window left open is a NERDTree 
" Stick this in your vimrc:
autocmd bufenter * if (winnr("$") == 1 && exists("b:NERDTree") && b:NERDTree.isTabTree()) | q | endif

" Use these variables in your vimrc. Note that below are default arrow
" symbols
let g:NERDTreeDirArrowExpandable = '▸'
let g:NERDTreeDirArrowCollapsible = '▾'

Plugin 'kana/vim-textobj-user'
Plugin 'rbonvall/vim-textobj-latex'

" Plugin 'tpope/vim-surround'
Plugin 'tpope/vim-surround'

" Plugin 'kien/ctrlp.vim'
Plugin 'kien/ctrlp.vim'

"Change the default mapping and the default command to invoke CtrlP:
  let g:ctrlp_map = 'C-p'
  let g:ctrlp_cmd = 'CtrlP'
  let g:ctrlp_user_command = 'find %s -type f'        " MacOSX/Linux

" Plugin 'altercation/vim-colors-solarized'
Plugin 'altercation/vim-colors-solarized'
syntax enable
set background=dark
colorscheme solarized

" Airlines
Plugin 'vim-airline/vim-airline'
Plugin 'vim-airline/vim-airline-themes'

" add the following to your vimrc to enable the extension:
let g:airline#extensions#tabline#enabled = 1

" Separators can be configured independently for the tabline, so here is how
" you can define "straight" tabs:
let g:airline#extensions#tabline#left_sep = ' '
let g:airline#extensions#tabline#left_alt_sep = '|'

" YouCompleteMe
" 
Plugin 'valloric/youcompleteme'

Plugin 'justinmk/vim-sneak'
let g:sneak#label = 1


" Unisnips
" 
" Track the engine.
Plugin 'SirVer/ultisnips'

" Snippets are separated from the engine. Add this if you want them:
Plugin 'honza/vim-snippets'

" Trigger configuration. Do not use <tab> if you use https://github.com/Valloric/YouCompleteMe.
let g:UltiSnipsExpandTrigger=",a"
let g:UltiSnipsJumpForwardTrigger=",b"
let g:UltiSnipsJumpBackwardTrigger=",z"




" If you want :UltiSnipsEdit to split your window.
let g:UltiSnipsEditSplit="vertical"

Plugin 'lervag/vimtex'
Plugin 'gerw/vim-latex-suite'

imap <F1>  <Plug>IMAP_JumpForward
set winaltkeys=no

"Plugin 'rbonvall/vim-textobj-latex'
"Plugin 'gibiansky/vim-latex-objects'
" easy align 
"
Plugin 'junegunn/vim-easy-align'
" Start interactive EasyAlign in visual mode (e.g. vipga)
xmap ga <Plug>(EasyAlign)

" Start interactive EasyAlign for a motion/text object (e.g. gaip)
nmap ga <Plug>(EasyAlign)


Plugin 'itchyny/calendar.vim'
let g:calendar_google_calendar = 1
let g:calendar_google_task = 1



Plugin 'vimtips.zip'

Plugin 'luochen1990/rainbow'
let g:rainbow_active = 1 "0 if you want to enable it later via :RainbowToggle

Plugin 'w0rp/ale'
Plugin 'vim-pandoc/vim-pandoc'
Plugin 'vim-pandoc/vim-pandoc-syntax' 
Plugin 'easymotion/vim-easymotion'

" map f <Leader> 

" map <Leader> <Plug>(easymotion-prefix)
" <Leader>f{char} to move to {char}
map  <Leader>f <Plug>(easymotion-bd-f)
nmap <Leader>f <Plug>(easymotion-overwin-f)

" s{char}{char} to move to {char}{char}
nmap s <Plug>(easymotion-overwin-f2)

" Move to line
map <Leader>L <Plug>(easymotion-bd-jk)
nmap <Leader>L <Plug>(easymotion-overwin-line)

" Move to word
map  <Leader>w <Plug>(easymotion-bd-w)
nmap <Leader>w <Plug>(easymotion-overwin-w)
""



Plugin 'xolox/vim-misc'
Plugin 'xolox/vim-notes'
"let g:notes_suffix = '.tex'

 Plugin 'vimoutliner/vimoutliner'

Plugin 'matchit.zip'

let b:match_words = 'if:end if'  "'\<big\>:\<big\>'
"let maplocalleader = '\'  " # Default if option is not set

let g:notes_directories = ['~/Dropbox/Shared Notes']

Plugin 'tpope/vim-repeat'


Plugin 'tpope/tpope-vim-abolish'

Plugin 'tpope/vim-commentary'



Plugin 'tpope/vim-speeddating'

"Plugin 'svermeulen/vim-easyclip'
"set clipboard=unnamed



" Plugin 'jceb/vim-orgmode'
filetype plugin indent on
"Plugin 'vimwiki/vimwiki'
call vundle#end()            " required
filetype plugin indent on    " required
set wrap 
" To ignore plugin indent changes, instead use:
"filetype plugin on
"
" Brief help
" :PluginList       - lists configured plugins
" :PluginInstall    - installs plugins; append `!` to update or just :PluginUpdate
" :PluginSearch foo - searches for foo; append `!` to refresh local cache
" :PluginClean      - confirms removal of unused plugins; append `!` to auto-approve removal
"
" see :h vundle for more details or wiki for FAQ
" Put your non-Plugin stuff after this line

if has("user_commands")
    command! -bang -nargs=? -complete=file E e<bang> <args>
    command! -bang -nargs=? -complete=file W w<bang> <args>
    command! -bang -nargs=? -complete=file Wq wq<bang> <args>
    command! -bang -nargs=? -complete=file WQ wq<bang> <args>
    command! -bang Wa wa<bang>
    command! -bang WA wa<bang>
    command! -bang Q q<bang>
    command! -bang QA qa<bang>
    command! -bang Qa qa<bang>
endif



syntax on
let g:loaded_vimballPlugin = 1
"so '~/.vim/bundle/Vimball/autoload/vimballPlugin.vim'

augroup nonvim
   au!
   au BufRead *.png,*.jpg,*.pdf,*.gif,*.xls*,*.ppt*,*.doc*,*.rtf sil exe "!open " . shellescape(expand("%:p")) | bd | let &ft=&ft
augroup end

" autocmd BufNewFile,BufReadPost *.md set filetype=tex
"
" type in the date
imap <F6> <C-R>=strftime("%Y-%m-%d")<CR>
